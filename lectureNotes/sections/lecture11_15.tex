\documentclass[../main.tex]{subfiles}

\begin{document}

\section{Lecture 11}{Eigenvalue Problems}


\begin{definition}
    ALgebraic multiplicity of an eigenvalue counts the number of times the eigenvalue occurs as a root of the characteristic equation. Geometric multiplicity counts how many linearly independent eigenvectors correspond to an eigenvalue. It is a proveable fact that 
    algebraic multiplicity is at least as much as the geometric multiplicity. If algebraic multiplicity is greater than geometric multiplicity, we say that the matrix is defective.
\end{definition}

\begin{theorem}
    Similar matrices share the same eigenvalues.
\end{theorem}

\begin{theorem}
    If a matrix is defective, then it cannot have an eigenvector basis.
\end{theorem}

\begin{proposition}
    A matrix is diagonalizable iff it has an eigenvector basis.
\end{proposition}

\begin{proposition}
    The following matrix transformations change eigenvalues and eigenvectors as follows:

    \begin{outline}
        \1 $A \rightarrow (A - \sigma I)$ causes $\lambda \rightarrow \lambda - \sigma$. 
        \1 $A \rightarrow \inv{A}$ causes $\lambda \to \frac{1}{\lambda}$>
        \1 $A \rightarrow A^k$ causes $\lambda \to \lambda^k$.
        \1 If $A = PX\inv{P}$, then $X$ has the same eigenvalues but every eigenvector $v$ now becomes $\inv{P}v$.
    \end{outline}
\end{proposition}

\begin{proposition}
    Suppose that we perturb a diagonal matrix $A$ with some matrix $E$. Then the distance between any eigenvalue $u$ of $A + E$ to an eigenvalue of $A$ $\lambda_k$ closest to $u$ is bounded by $k(A) \norm{E}$.
\end{proposition}

\section{Lecture 12}

\begin{problem}
    Characteristic Polynomial
For which of the following reasons is the characteristic polynomial of a matrix NOT useful, in general, for computing the eigenvalues of the matrix?

Select all that apply:
Its coefficients may not be well determined numerically.
Its roots may be difficult to compute.
Its roots may be sensitive to perturbations in the coefficients.
None of these
\end{problem}

\begin{solution}
    Yes, the roots change if the coefficients change -- hence any perturbation will affect the roots; determining the coefficients numerically is also problematic, simply because numerical comptutation requires rounding error and truncation error. The second is a given.
\end{solution}

\begin{problem}
    Problem Transformations and Spectral Radius
Which of the following transformations preserve the spectral radius of a matrix A?

Select all that apply:
Powers
None of these
Shift
Polynomial
Inversion
\end{problem}
\begin{solution}
    Obviously, none of them, since they all change the eigenvalues.
\end{solution}

\begin{problem}
    Diagonalizability
Of the classes of n×n matrices listed below, which is the smallest class of matrices that are not necessarily diagonalizable by a similarity transformation?

Choice*
normal matrices
all matrices
real symmetric matrices
matrices with n distinct eigenvalues
\end{problem}
\begin{solution}
    A spectral theorem asserts that normal matrices are unitarily diagonalizable; a theorem asserts that real symmetric matrices have an orthogonal basis -- hence they are diagonalizable; in general, any matrix with an eigenbasis is diagonalizable.
\end{solution}

\begin{problem}
    Let $A = XD\inv{X}$. Suppose $\hat{A} = A + \delta A = \hat{X} \hat{D} \inv{(\hat{X})}$. Which matrix is $\hat{A}$ similar to?
\end{problem}

\begin{solution}
    $\inv{X}\hat{A} X = \inv{X}AX + \inv{X}\delta A X = D + \inv{X} \delta A X$
\end{solution}
\end{document}










