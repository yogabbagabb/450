\documentclass[../main.tex]{subfiles}



\begin{document}

\section{Lecture 25}
\begin{definition}
    An ordinary differential equation is a differential equation where the state vector
    depends only on one variable, usually time. A partial differential equation has the
    state vector depend on several variables.
\end{definition}

\begin{definition}
    A partial differential equation is in implicit form, if for each derivative of the state
    vector $y$ that we denote by $y^k$, we have

    \[
        f(t, y,  y^2, \dots y^k) = 0
    \]

    By contrast, an ODE is said to be in explicit form is we can write $y^k$ as

    \[
        f(t, y', y'', \dots ,y^{k-1})
    \]

    
\end{definition}

\begin{definition}
    The order of an ODE is the degree of the highest derivative that occurs in the implicit
    form representation of an ODE.
\end{definition}

\begin{remark}
    We will restrict  our attention to first order ODEs, for suppose we are given an explicit ODE, then we can define $u_k = y^{k} = f(t, y, y'', \dots y^{k-1})$ and we will find that

    \begin{align*}
        \begin{bmatrix}
            u_2 \\
            u_3 \\
            u_4 \\
        \end{bmatrix}
        =
        \begin{bmatrix}
            u_1' \\
            u_2' \\
            u_3' \\
        \end{bmatrix}
    \end{align*}
    which is a system of $kn$ first order equations (here, realize that $u_k$ is a vector in $\R^n$).
\end{remark}

\begin{remark}
    We can numerically solve for an ordinary differential equation $y' = f(t,y)$ by first making the following approximation:

    \begin{align*}
        \text{Let} \\
        y_0 = \text{ The initial position} \\
        t_0 = \text{ The initial time} \\
        h_0 = \text{ The distance between $t_0$ and $t_1$} \\
        \intertext{In some sense, the slope at time $t_0$ is given by} \\
        f(t_0, y_0) \\
        \text{Approximate} \\
        y_1 = y_0 + h_0f(t_0, y_0) \\
    \end{align*}


    Note that this approximation can also be understood as a difference quotient:

    \[
        \frac{y_{k+1} - y_{k}}{h_k} = f(t_k, y_k)
    \]
\end{remark}

\begin{remark}
    The equation $y' = f(y,t)$ by itself doe snot completely specify the solution $y$. For the ODE merely specifies the slope of a solution and, thus, there are in principle an infinite number of solutions to it. We can often constrain the class of solutions to a unique solution, however, if we additionally maintain the solution must satisfy an initial condition $(t_0, y_0)$. Note, however, that this is merely a necessary condition and not a sufficient one for the solution to an ODE to be unique..
\end{remark}<++>
\begin{definition}
    Suppose that we are presented with the ODE $y' = f(t,y)$ with intiial conditions $(t_0, y_0)$. SUppose that $\hat{y}(t)$ is a solution to the ODE. Then we say that $\hat{y}(t)$ is a stable solution if it holds that for every $\e$ there exists some $\delta$ such that whenever $\abs{\hat{y}(t_0) - \tilde{y}(t_0)} \leq \delta$ where $\tilde{y}$ is some other solution, then $\abs{\tilde{y}(t) - \hat{y}(t)} \leq \e$ for all $t \geq t_0$. We say that $\tilde{y}$ is asymptotically stable if $\abs{\tilde{y}(t) - \hat{y}(t)} \leq 0$ as $t \to \infty$.
\end{definition}<++>
\section{Lecture 26}


\begin{definition}
    Let $t_k$ be some time at iteration $k$. Let $y_k$ be the computed solution at $t_k$. Let $y(t_k)$ be the true solution of hte ODE (given initial conditions $y_0, t_0$). Then

    global error is the error that we've made in an unforgiving sense: it $e_k \coloneq y_k - y(t_k)$. \\

    By contrast, local error is the error that we make at iteration $k$ assuming that our guess at iteration $k-1$ was accurate. That is, we let $l_k \coloneq y_k - u_{k-t}(t_k)$ where $u_k$ is the solution to the ODE with initial conditions given by $(t_{k-1}, y_{k-1})$.
\end{definition}

\section{Lecture 28}


\begin{definition}

    A boundary value problem is of the form

    \[
        u'(x) = f(u(x))
    \]

    and $g(u(a), u(b)) = 0$ is the boundary condition.

    where $u: [a,b] \to \R^n$ 
\end{definition}


\begin{remark}
    A boundary value problem has a solution (it is hinted in lecture) if the IVP
    \[
        u'(x) = f(u(x))
    \]

    with initial condition $(t_0, e_i)$

    is solveable for all unit vectors.
    
\end{remark}

\begin{definition}
    The shooting method does the following:

    Suppose that we're handed 

    \begin{align}
        u''(x) = f(u(x)) \\
        u(a) = \text{ some vector } \\
        u(b) = \text{ some vector } \\
    \end{align}

    Then we will set $u'(a)$ to a random value and check to see whether
    the resulting IVP gives a solution that matches the value of $u(b)$ above.
\end{definition}

\end{document}







