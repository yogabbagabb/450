\documentclass[a4paper]{article}
\usepackage{preamble}
\usepackage{374_preamble}


\begin{document}
Name: Aahan Agrawal \\

Date: April 10th, 2019 at 2:30 pm \\

What: The Cozad Competition Finalists Presentation \\

I attended the cozad finals event on Thursday, April 10th. Specifically, I saw finalists present their ventures. I gathered the following impressions (they are naiive, because I have never done anything entrepeneurial (including learning about the subject) prior to this semester), observing all of them:

\begin{outline}
    \1 Expectedly, most of the ventures were introducing third wave technologies. That is, no technology was so innovative that it seemed like it might disrupt an existing industry or else introduce an entirely new industry. Even so, there were people (likely investors?) who kept asking pointed questions about these ventures, questioning their feasibility: ie there seems to be a recognition on both the part of investors and venture founders that most innovations will be incremental.
    \1 Many of these innovations tended to help niche customers in ways that I found frivolous (a device that helps reptile owners take care of their reptiles when they are gone). Being willing to target niche groups is, however, important if you want to launch something, because moost ventures are making marginal changes to niche industries.
\end{outline}

\breathe \\
Now I review what $1$ of the finalists discussed:

DeepWalk is a venture that aims to improve the process of assembling ramps for people with disabilities. For example, wheelchair users need to use ramps. The process of erecting a ramp in a city is tiered process: the city must first determine that a need exists, it must then contact engineering consultants who need to design an adequate ramp; it must then contact surveyors who will build the ramp. Coupling this process with the occassional litigation that someone with a disability or some organization may throw at a city because of lack of accessbility -- and we find that cities pay large costs on ramp related matters. \\

DeepWalk is a team that has designed an app (also called DeepWalk) that allows a user to take a series of photos through their smartphone of an area that may need a ramp. The app then constructs a schematic or offers some data that is suitable either for review by engineering consultants or for construction by surveyors. This app will streamline the process of designing ramps: \\

Ordinary people can now take photos of areas that they think warrant ramps; a design of a ramp or some data with which one can make a ramp can now be forwarded to engineering consultants (eliminating any overhead of a government body contacting these consultants) -- these consultants can then approve or reject designs -- and then surveyors can take the approved designs and (subject to approval of a city or some other group) make a ramp. \\
 
I did not completely follow the revenue scheme for this app. It seems to have the following shape: \\

\begin{outline}
    \1 A city will pay DeepMind to upload scans of areas.
    \1 Surveyors will pay DeepMind to have the opportunity to view scans of areas and thereafter bill cities for their review work.
    \1 Contractors will pay DeepMind to use approved diagrams and then bill cities for their review work.
\end{outline}

The founder of this initiative mentioned that while this app would largely be used by city groups, its reach was not limited to them and any user wanting a ramp (for some reason) could use it. \\

He mentioned that cities would be particular targets for this app, however, because of their fears for litigation; ie they would rather use this app liberally and prevent an otherwise costly litigation for one slight mishap. \\

%\breathe \\

%Verge Products is a venture that aims to introduce an automatic reptile feeding device. This device discharges healthy crickets (even when a homeowner is not available to discharge them) to reptiles that eat these crickets. The market for this product is somewhat large:; there are 9.4 million reptiles in the US, corresponding to a 1.4 billion market. While there is potential for this project, the project founder explained that funding for this project cannot be crowd funded because it targets such a niche industry. That is, paradoxically, the market is large enough, in spite of being a niche market, for sale of this product yet it is not so large that it can be crowdfunded for its initial market entry. \\

%I learned about a few advertising tricks that I was not hitherto aware of. THe first is that depending on what you check on facebook or like, you are shown targeted ads relating to that object of interest. The founder of Verge products used this trend to explain that he could selectively target reptile lovers using this advertising, in spite of reptile lovers being a sparse segment of the population. There are also so called social media influencers who can sway public opinion on a product -- the owner of Verge Products has explained that he has made contact with several of these influencers, and they all think that his project can be successful, giving hope to the owner that there is good potential to reach out to people. \\

%\breathe \\

%Daniat intends to democratize the sorting of dates (the food that we can eat).
%Now why do dates require sorting? It turns out that some dates are more sweet than others and, hence, less suitable for people avoiding sugar like diabetes patients; some dates are also just inedible (or undeveloped or of inferior quality). \\

%At present, a portion of the date market works as follows: small time farmers and harvesters collect dates manually -- they then send them to large corporations that sort dates and sell them. Daniat has developed a device (costing about \$5000 by their estimates) that can, with the same efficiency as three workers, sort dates. \\

%There are over 1 million date plants in the US; each machine costs \$5000 -- which is within the anecdotal \$10,000 that most farmers would be willing to spend on such equipment -- and so there is significant potential for this product. The fact that the sorting equipment that large corporations use is over \$100,000 also means that there is limited room for small time farmers to make the jump to large time sorting. \\








\end{document}



