Daymond John is an entrepeneur who started a clothing company for males. Its total value is around 6 billion; the company is called FUBU. On Tuesday, he discussed his entrepeneur story at the Illini Union. These are a few of the things that I learned. These observations are all related to the culture of entrepeneurship rather than any observation that may improve one's entrepeneurail ability. 

He grew up in an environment in which it seemed that small business owners were benefactors and noble members of society. According to John, the business owners put only blame on themselves an d success on eveyone else; moreover, they support their community by giving it employment, even if it means not making anything themselves. I make this observation, because I grew up in an environment that did not portray small business owners this way; on the contrary, I grew up for some time believing that the business owners I had encountered were purely motivated out of profit, power and other shallow motivations. For you see, I grew up in what I would describe as the little India of Lagos, Nigeria. In Lagos, the few people whom I met that were business owners had noticeable wealth that far exceeded the wealth of the other, working Indians that I knew. These business owners were the people who would vacation abroad quite frequently, while the other, working Indians would be scarcely return to India, forget travel abroad, for about 3 weeks in a year. There was also an element of unapproachability that surrounded some of these business-owning Indians, I felt. Combine these observations with the observations of a few cynics in my family and you grow to feel that business owners are not really benefactors of society. 

I make this observation, because I really did witness a different cultural take on the regard for businessmen and entrepeneurs. I did not expect a childhood view that looked upon them in a societally helpful way.

Daymond's entrepenurial success is also the result of the hip-hop that co-existed in his chilhood. From the following observation, which he presented on Tuesday, I have learned how he framed an otherwise negative circumstance into a positive one in his life. He remarked that the majority of hip hop stars during his childhood hapenned to originate from about a 5 mile radius of his home in Queens. A non-trivial number of hip-hop stars were from Queens. It seemed to Daymond, however, that upon becoming successful, these hip hop stars would take to wasteful habits, splurging their newly acquired wealth. In other words, hip hop seemed only a means of either fun or served a one stop shop to fame and extravagance and offered no good besides that. Daymond saw, however, the potential for hip hop clothing -- he saw the interest in it and capitalized on that by forming FUBU. In short, he took an otherwise meaningless childhood phenomenon and worked it to his favor. 



\breathe \\
After his mother and father divorced, he had to start working for several others

He describes that this generation has the ability to cater products to people -- there
is the story of someone who sells custom made t shirts. 

He describes his initiative when he was young: absent from his discussion is any mention about his work


There was a dearth of good influences in Queens, where he grew up -- he saw pele who were earning money quite a bit -- hip hop was something formative for the guy -- that amksked an impression.

he got some work at the colloseum -- looks at their selflessness and learns from them.
https://www.inc.com/magazine/201804/emily-canal/daymond-john-shark-tank-goals-rise-grind.html


Da
